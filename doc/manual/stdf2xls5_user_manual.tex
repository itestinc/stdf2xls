\documentclass[letterpaper]{article}
\usepackage[pdftex]{graphicx}
\usepackage{tp}
\usepackage{unitsdef}
\usepackage{pifont}
\usepackage{sectsty}
\usepackage{amsmath}
\usepackage{eso-pic}
\usepackage[T1]{fontenc}
\usepackage{varioref}
\usepackage{caption}
\usepackage{textcomp}
\usepackage{float}
\usepackage{epsfig}
\usepackage{tikz}
\usetikzlibrary{arrows}
%%\usepackage{ulem}
\pagestyle{plain}
\renewcommand{\arraystretch}{1.4}
\begin{document}
\usefont{T1}{ua1}{m}{n}\selectfont
\newcommand{\tfont}{\usefont{T1}{ua1}{m}{n}\selectfont\footnotesize}
\newcommand{\bfont}{\usefont{T1}{ua1}{b}{n}\selectfont\tiny}
\newcommand{\xfont}{\usefont{T1}{ua1}{m}{n}\selectfont\scriptsize}
\newcommand{\lfont}{\usefont{T1}{ua1}{m}{n}\selectfont\large}
\renewcommand{\captionfont}{\it }
\renewcommand{\date}{February 26, 2020}
\newcommand{\ver}{V1.0}
\newcommand{\tablecap}{\hline\end{tabular}\end{table}\end{center}}
\renewcommand{\versionhistory}{
\vspace*{1in}
\begin{center}
\begin{table}[H]\caption*{Revision History}
\centering
\xfont\begin{tabular}[H]{|c|c|c|c|}
\hline
{\bf Version} & {\bf Author} & {\bf Date} & {\bf Changes}\\
\hline
\hline
1.0 & Eric West & 2/26/20 & Initial Release \\
\hline
\end{tabular}
\end{table}
\end{center}
}
\maketitle
\setcounter{tocdepth}{2}
\tableofcontents
\clearpage
\makebg

\section{Introduction}
stdf2xls 5.0 is a program that converts STDF files into spreadsheets, wafermaps, and/or histograms.
It is natively compiled from the D-language and therefore it uses much less memory than the previous
stdf2xls 4.0 java program, and is significantly faster too.  It has many new features:
\begin{itemize}
\item Spreadsheet and ASCII wafermaps
\item Spreadsheet histograms
\item Ability to display hex, integer, and string data values in the spreadsheet
\item New algorithm correctly orders tests even if the testflow varies from device to device
\item Spreadsheet colors and fonts are now customizable
\item Improved spreadsheet layout
\item Many ways to sort and order device data
\item Ability to modifiy any STDF text field with regular expressions
\item Ability to write out STDF files
\item Multiple different devices types can be processed simultaneously
\end{itemize}

\section{Getting the program}

The source code is available on github. The source code can be downloaded
from gitlab with git, or just the exececutable for Windows or GNU/Linux
may be downloaded from the gitlab web page.  This manual may also be copied
from the gitlab webpage.\\
\\
To download the software with git use the following command:
\begin{verbatim}
git clone https://github.com/itestinc/stdf2xls.git
\end{verbatim}

To compile the source code on GNU/Linux you also need dub, gcc and dmd (dmd is the D compiler).

\section{Installation}

To just install the executable, the program can be downloaded from github.com.

\subsection{GNU/Linux Installation}

For GNU/Linux go to https://github.com/itestinc/stdf2xls, click on the dist folder, then
click on the stdf2xls file, then press the Download button.  After the file has downloaded
put it in a directory that in in your executable search path.

\subsection{Windows Installation}

For Windows go to https://github.com/itestinc/stdf2xls, click on the dist folder, then 
click on the stdf2xls.exe file, then press the Download button.  After the file has downloaded
put it in a directory that in in your executable search path.

\section{Usage}

A summary of the command line options can be printed by running the program
with the '--help' option.  More detailed information about the command line
options is given in this manual.

There are nine primary operations that can be done with the program, and they may be
done individually or simultaneously:
\begin{itemize}
\item -s or --genSpreadsheets will generate spreadsheets for the STDF measurement data
\item -h or --genHistograms will generate histograms for the STDF measurement data
\item -w or --gnWafermaps will generate wafermaps for the STDF bin data
\item -d or --dumptext will generate an ASCII dump of the STDF file(s)
\item -b or --dumpBytes will generate an ASCII dump of all of the bytes in the STDF file(s)
\item -D or --digest will dump each unique set of header information found in the STDF files(s)
\item -m or --modify <record field fromRegex : toRegex> will modify any text field in any record type 
\item -g or --generateRCFile
\item -o or --outputDir <folder\_name> will cause the STDF files that are read in to be written out to this folder
\end{itemize}

Each of these major options have several more options to refine their behavior.

\subsection{Generating a Spreadsheet}
\begin{verbatim}
java [-mx<M>] -jar <path_to_stdf2xls4_jar_file> [options] -x <spreadsheet_file> <stdf_files>
\end{verbatim}
\begin{itemize}
\item <M> is the amount of memory to allocate to the java virtual machine for example, 4G, 500M, 12G.
\item [options] are explained in the next section, and are optional.
\item The spreadsheet\_file must have a suffix of ".xls" or ".xlsx".  The spreadsheet format is
determined from this suffix.
\item There can be one STDF file or many STDF files specified; also "*.std" or "*.stdf" will work.
\end{itemize}

\subsection{Converting STDF to ASCII}
\begin{verbatim}
java -jar <path_to_stdf2xls4_jar_file> -d <stdf_files>
\end{verbatim}

\subsection{Command Line Options}
\begin{verbatim}
java -jar <path_to_stdf2xls4_jar_file> --help
\end{verbatim}

\section{Command Line Options}
Options can be specified as a single dash followed by a single character, or a double dash
followed by multiple characters.
\begin{itemize}
\item -a [<char>] or --pin-suffix [<char>]\\
This option tells the program that if a test name of a ParametricTestRecord has
a character <char> in it then the characters following <char> are to be interpreted
as a pin name for that test.  If <char> is not specified, then by default '@' will be used.
The pin name is displayed on a separate row or column in the test header.
\item -b or --one-page\\
Normally STDF files from different wafers or test steps are displayed in different spreadsheets.
This option will combine them all on one page with an additional row or column for the test
step or wafer number.
\item -c or --columns\\
Specifying this option will increase the maximum number of columns in the spreadsheet
from 1024 to 16384.  Note that with this option the spreadsheet will not be
readable with Libreoffice.
\item -d or --dump\\
This option causes the STDF data to be dumped in ASCII format to STDOUT.
\item -g or --gui\\
This option will run the program from a GUI instead of the command line.  This option
is experimental and will probably be deprecated in a future version.
\item -h or --help\\
Prints a summary of the command line options.
\item -j or --jxl-xls-name\\
This option specified to use the JExcel library instead of the Apache POI library
for spreadsheet generation.  Sometimes the Apache library for "xls" format is problematic.
The JExcel library can be tried to see if it resolves any formatting problems.  Note
that this option has no effect when using "xlsx" or XML format, and XML format is
recommended anyways.
\item -l <file> or --logo <file>\\
This option specifies a logo (PNG file) to include in the upper left
corner of the spreadsheet.  The logo should have an aspect ratio of
about 290(W) to 120(H).  It will be scaled to fit within the required area.
\item -m or --modifier\\
This is an experimental feature that can modify STDF fields on the fly as the STDF is loaded.
Currently it is only enabled for DatalogTextRecords. This option has the form of:\\
-m "R:Record\_t F:FieldDescriptor C:Condition\_t V:oldValue N:newValue"\\
Where record type currently can only be DTR, the field descriptor can only be TEXT\_DAT, and
Condition\_t can be EQUALS, CONTAINS, or TRUE.  Also, multiple modifiers may be specified by using multiple -m options.
This implementation is lame, and will probably later be replaced by a regular expression engine.\\
\begin{verbatim}
-m "R:DTR F:TEXT_DAT C:CONTAINS V:2.00 N:2.0"
\end{verbatim}
This will check every DatalogTextRecord for the string "2.00", and
if found, it will replace it with "2.0".  Currently whitespace in the
string cannot be handled.  The Condition TRUE is currently useless. The EQUALS
condition requires that the entire text field be equal to the oldValue for
a replacement to occur.
\item -n or --no-wrap-test-names\\
When the the test names run horizontally across the spreadsheet they will be wrapped
to keep the columns from getting too wide.  This option will prevent wrapping the test
names, but can give very wide columns.
\item -o or --no-overwrite\\
Currently this option is not working.
\item -p <integer> or --precision <integer>\\
This option specifies how many digits will be to the right of the decimal
point for floating point values.  The default is 3.
\item -r or --rotate\\
This option transposes the spreadsheet.  By default test names run horizontally across
the spreadsheet and devices run vertically down the spreadsheet. With the -r option
the test names will run vertically down the spreadsheet and devices will run horizontally
across the spreadsheet.
\item -s or --sort\\
this options sorts the data by serial number or XY-coordinate (for wafersort).
\item -v or --dont-skip-search-fails\\
Verigy (AKA Advantest) testers can output bogus test data as tests related to searches.  By default
these will be ignored.  This option will include them in the spreadsheet.
\item -x <file.xls[x]> or --xls-name <file.xlsx[x]>\\
This options specifies the name of the spreadsheet file and is required (unless using the -d option).
If the suffix of the filename is ".xlsx" then the newer XML format will be used for the spreadsheet.
Otherwise the older ".xls" format will be used.  In the future the xls format may be deprecated.
\item -y or --dynamic limits\\
If this option is used, then the high and low limits of all parametric tests are
searched, and if they vary by more than 0.1\percent then the corresponding test will use dynamic limit
which means that the column to the left of the test will be the lower limit used and the column to
the right of the test will be the upper limit used.
\end{itemize}
\clearpage
\section{Support for Alphanumeric Serial Numbers}
For alpha-numeric serial numbers, print a text field to the datalog
at the beginning of the test flow that looks like this:

\begin{verbatim}
TEXT_DATA: S/N : <serial_number>
\end{verbatim}
The strings "TEXT\_DATA" and "S/N" must match exactly what is shown here.  Whitespace
around the colons is ignored.  The maximum length of any string printed to the datalogger
must not exceed 255 characters.

\section{Custom Spreadsheet Data Values}
Arbitrary data for each device may be mapped to the spreadsheet as a test result along
with the rest of the test data.  To get this information into the STDF file, print a 
text field to the datalogger using the format:
\begin{verbatim}
TEXT_DATA : test_name : value [(units)] [ : test_number [ : site_number [ : head_number ]]]
\end{verbatim}
All fields after the value are optional.  Whitespace around the colons is ignored.

\section{Spreadsheet Header Fields}
It is often desirable to include header information in each spreadsheet.  The header information
can be encoded into datalog text fields that are printed out for each device.  Note that it is
expected that the header data be constant for an entire test session.  If the header information
changes within a set of STDF files, then there will be multiple spreadsheets generated (within a workbook)
with each spreadsheet having different header information.

To generate a header field, print the field to the datalogger using this format:
\begin{verbatim}
>>> <header_name> : <header_value>
\end{verbatim}
The prefix "\gT\gT\gT" indicates a header field.  The header name can be any string (not containing a colon)
and the header value can be any string.
\clearpage
\subsection{Legacy Header Fields}
There were a set of fixed header fields that are still supported, but are now deprecated (don't use them on new test programs).
These field were simply a header name followed by a header value.  The recognized header names
are:
\begin{itemize}
\item CUSTOMER:
\item DEVICE NUMBER:
\item SOW:
\item CUSTOMER PO\#:
\item TESTER:
\item TEST PROGRAM:
\item LOADBOARD:
\item CONTROL SERIAL \#s:
\item JOB \#:
\item LOT \#:
\item STEP \#:
\item TEMPERATURE:
\end{itemize}

These header names may be combined with the new header format if necessary.

\section{Modifying STDF Text fields}

\section{Wafermaps}


\section{Histograms}


\section{Look and Feel Customization}



\end{document}
